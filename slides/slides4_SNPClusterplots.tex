\documentclass{beamer}
\usetheme{Boadilla}
\usepackage{graphicx} % Required for inserting images
\usepackage{ngerman}
\usepackage{hyperref}

\title{Funktionelle Genomanalysen 2023}
\subtitle{Asynchrone Übung zu SNP-Clusterplots}
\author{Dr. Janne Pott}
\date{09.-11. Juni 2023}

\begin{document}

\begin{frame}
\titlepage
\end{frame}

\begin{frame}{Allgemeines}
    \begin{itemize}
        \item Array-abhängig werden die verschiedenen Allele mit Labels versehen (farbcodiert)
        \item Pro Sample und SNP werden die Intensitäten beider Labels bestimmt
        \item Jeder Plot ist SNP-spezifisch
        \item Jeder Punkt im Plot entspricht einem Sample
        \item X-Achse ist die log-transformierte Ratio der Intensitäten: $log_2(A/B)$
        \item Y-Achse ist das log-transformierte Produkt der Intesitäten: $0.5 \cdot log_2(AB)$
        \item AB Cluster sollte über den anderen beiden liegen, weil hier doppelte Intesität gemessen
    \end{itemize}
\end{frame}

\begin{frame}{Clusterdefinition im Idealfall}
    \begin{itemize}
        \item Genotyp AA
        \begin{itemize}
            \item hohe A Intesität, niedrige B Intesität 
            \item Ratio ist größer als 1
            \item Log-Transformierte Ratio ist größer als 0
            \item \textbf{blaues Cluster} 
        \end{itemize}
        \item Genotyp BB
        \begin{itemize}
            \item hohe B Intesität, niedrige A Intesität 
            \item Ratio ist kleiner als 1
            \item Log-Transformierte Ratio ist kleiner als 0
            \item \textbf{grünes Cluster} 
        \end{itemize}
        \item Genotyp AB
        \begin{itemize}
            \item mittlere A Intesität, mittlere B Intesität 
            \item Ratio ist in etwa 1
            \item Log-Transformierte Ratio ist in etwa 0
            \item \textbf{gelbes Cluster} 
        \end{itemize}
    \end{itemize}
    
\end{frame}
\begin{frame}{SNP Kriterien}
\begin{table}[]
    \centering
    \begin{tabular}{l|l|r}
    Kriterium & Bedeutung & Threshold \\   
    \hline
    Call Rate & 1 – Anteil missings            &  $<97\%$  \\
    p(HWE)    & p-Wert des exakten Fisher Test & $<1x10^{-6}$ \\
    p(PA)     & p-Wert des Chi-Quadrat Test    & $<1x10^{-7}$  \\
              &  der AF pro Platte             & \\
    MAC       & Minor Allele Count             & $<2$ \\
    \hline
    FLD       & Minimaler Abstand zwischen den & $<3.6$  \\
              & Cluster (bzgl. X-Achse)        & \\
    HetSO     & Abstand des AB-Clusters zu AA  & $<-0.1$    \\
              &  bzw. BB (bzgl. Y-Achse)       & \\
    HomRO     &  Verteilung der Cluster        & 3 Cluster: $<-0.9$ \\
              &  (bzgl. 0 der X-Achse)         & 2 Cluster: $<0.3$  \\
              &                                & 1 Cluster: $<0.6$  
    \end{tabular}
    \caption{Übersicht von relevanten SNP Kriterien. Die ersten vier sind Array-unspezifisch, die letzten drei sind spezifisch für Affymetrix Arrays. Es gilt: SNP filtern, wenn Wert kleiner als Threshold ist}
    \label{tab:my_label}
\end{table}
\end{frame}

\begin{frame}{Plattform-unspezifische Kriterien}
    \begin{itemize}
        \item \textbf{Call Rate}: s. HWE Aufgabe in erster Übung - es konnten nicht für alle Samples die Genotypen bestimmt werden
        \item \textbf{HWE}: s. HWE Aufgabe in erster Übung - in der Übung zu $\alpha = 0.5$ getestet, jetzt auf multiples Testen korrigiert
        \item \textbf{PA}: Plattenassoziation, Test ob es Batcheffekte pro Platte gibt. Ebenfalls auf multiples Testen korrigiert
        \item \textbf{MAC}: wenn das selterne Allel nicht mindestens 2 mal vorkommt, dann ist das quasi kein SNP mehr, weil fast alle homozygot sind und nur eine Person heterozygot. Damit kann man nicht vernünftig rechnen
    \end{itemize}
\end{frame}

\begin{frame}{FLD: Fisher’s Linear Discriminant}
\begin{figure}
    \centering
    \includegraphics[width=0.75\textwidth]{_figures/Slides4_FLD.JPG}
    \caption{Links FLD ok, rechts FLD kleiner als 3.6 (zwischen AB und AA) $\rightarrow$ SNP muss gefiltert werden}
    \label{fig:FLD}
\end{figure}
\end{frame}

\begin{frame}{HetSO: Heterozygous Strength Offset}
\begin{figure}
    \centering
    \includegraphics[width=0.75\textwidth]{_figures/Slides4_HetSO.JPG}
    \caption{Links HetSO ok, rechts HetSO kleiner als -0.1 (Mittelpunkt AB deutlich unter der Verbindungslinie zwischen AA und BB) $\rightarrow$ SNP muss gefiltert werden}
    \label{fig:HetSO}
\end{figure}
\end{frame}

\begin{frame}{HomRO: Homozygote Ratio Offset}
\begin{figure}
    \centering
    \includegraphics[width=0.75\textwidth]{_figures/Slides4_HomRO.JPG}
    \caption{Links HomRO ok, rechts HomRO kleiner als -0.9 (BB Cluster deutlich im positiven Bereich der x-Achse) $\rightarrow$ SNP muss gefiltert werden}
    \label{fig:HomRO}
\end{figure}
\end{frame}

\end{document}