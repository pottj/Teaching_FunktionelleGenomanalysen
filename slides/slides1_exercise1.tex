\documentclass{beamer}
\usetheme{Boadilla}
\usepackage{graphicx} % Required for inserting images
\usepackage{german}

\title{Funktionelle Genomanalysen 2023}
\subtitle{Übung 1: Grundlagen der genetischen Statistik}
\author{Dr. Janne Pott}
\date{09.-11. Juni 2023}

\begin{document}

\begin{frame}
\titlepage
\end{frame}

\section{Introduction}

\begin{frame}{Vorstellung}
\begin{itemize}
    \item Name
    \item Fachrichtung 
    \item Standort
    \item Erwartung an Übung
\end{itemize}
\end{frame}

\begin{frame}{Vorstellung}
\begin{itemize}
    \item Name: Janne Pott
    \item Fachrichtung 
    \begin{itemize}
        \item Genetische Statistik im Allgemeinen
        \item Entwicklung neuer kausalen Methoden unter Verwendung von Genetikdaten im Speziellen (Stichwort \textbf{Mendelische Randomisierung})
    \end{itemize}
    \item Standort: MRC BSU, University of Cambridge, UK 
    \item Erwartung an Übung
    \begin{itemize}
        \item Veranschaulichung einiger Konzepte
        \item Hinweise auf praktische Anwendung 
        \item Kontakt zu Medizinern
    \end{itemize}
\end{itemize}    
\end{frame}

\begin{frame}{Hinweise zu Moodle}
\begin{itemize}
    \item Alle relevanten Unterlagen stehen auf Moodle zur Verfügung
    \begin{itemize}
        \item Übungsaufgaben
        \item Folien
        \item Video (vermutlich)
    \end{itemize}
    \item Asynchrones Lernen: Multiple Choice Aufgaben zu
    \begin{itemize}
        \item SNP-Clusterplots
        \item GWAS-Plots
        \item ...
    \end{itemize}
    \item Forum: bitte Nutzen, andere haben evtl. die gleiche Frage!
\end{itemize} 
\end{frame}

\begin{frame}{Hinweise zur Übung}
\begin{itemize}
    \item Die Aufgaben werden in der Übung gemeinsam erarbeitet, daher bitte \textbf{Kamera an}. 
    \item Zur Lösung von manchen Aufgaben wird ein Taschenrechner o.ä. benötigt. 
    \item Am Ende des Moduls wird eine Musterlösung bereitgestellt. 
\end{itemize}
\end{frame}

\section{Exercise 0}

\begin{frame}{Präsenzaufgabe 1}
Definieren Sie folgende Begriffe

\begin{itemize}
    \item SNP
    \item nicht-synonyme Mutation
    \item Frameshift-Mutation
\end{itemize}
\end{frame}

\begin{frame}{Präsenzaufgabe 1 - Lösung}

\begin{itemize}
    \item SNP: single nucleotid polymorphism = Einzelnukleotid Polymorphismus = Punktmutation
    \begin{itemize}
        \item Variation eines Basenpaares an einer Stelle im Genom
        \item Bsp.: SNP in mcm6 führt zu Laktoseintoleranz
    \end{itemize}
    \item nicht-synonyme Mutation: SNP im codierenden Bereich eines Gens, der dazu führt, dass eine andere Aminosäure eingebaut wird.
    \begin{itemize}
        \item betrifft nur eine Aminosäure
    \end{itemize}
    \item Frameshift-Mutation: SNP im codierenden Bereich eines Gens, der dazu führt, dass das Leseraster der RNA-Polymerase sich ändert. 
        \begin{itemize}
        \item betrifft alle folgenden Aminosäure bzw Länge des Proteins
    \end{itemize}
\end{itemize}
\end{frame}

\section{Exercise 1}

\begin{frame}{Aufgabe 1: Crossing-over \& LD (1)}
\begin{enumerate}
    \item Definieren Sie anhand der Abbildung \ref{fig:CrossingOver} den Begriff \textbf{Crossing-over}.
    \item Erläutern Sie den Zusammenhang zwischen der \textbf{Crossing-over} und \textbf{LD-Struktur} des Genoms.
    \item Betrachten Sie Tabelle \ref{tab:4FT}. Bestimmen Sie die \textbf{Randverteilungen} und berechnen Sie das \textbf{LD-Maß $r^2$}! Formel: $$r^2 = \frac{(p_{00}p_{11} - p_{01}p_{10})^2}{p_{0.}p_{.0}p_{1.}p_{.1}}$$
    \item Interpretieren Sie das Ergebnis! Was sind die \textbf{häufigen Haplotypen}? Was bedeutet dies für ein doppelt heterozygotes Individuum?
    \item Würden Sie zwischen SNP 1 und SNP 3 ein höheres oder niedrigeres $r^2$ erwarten? Begründen Sie Ihre Entscheidung!
\end{enumerate} 
\end{frame}

\begin{frame}{Aufgabe 1: Crossing-over \& LD (1)}
\begin{figure}[h]
\begin{center}
\includegraphics[width=0.5\textwidth]{../figures/Exercise11_CrossingOver_mod.jpg}
\caption{Crossing-over eines Chromosoms. Modifiziert aus Alberts et al., Molecular Biology of the Cell, 2008}
\label{fig:CrossingOver}
\end{center}
\end{figure}
\end{frame}

\begin{frame}{Aufgabe 1: Lösung (1)}

\textbf{Crossing-over}: Austausch von Teilen homologer Chromosomen während der Meiose $\longrightarrow$ Rekombination, Mischung der Erb-Informationen der Eltern

\begin{itemize}
    \item Step 1: DNA Replikation der väterlichen und mütterlichen DNA
    \item Step 2: Alignierung der duplizierten homologen Chromosomen
    \item z.T. Interaktion zwischen komplementären DNA-Sequenzen 
\end{itemize}
\end{frame}

\begin{frame}{Aufgabe 1: Lösung (2)}

\textbf{Haploblöcke} wechseln sich mit \textbf{Rekombinations-Hotspots} ab. Dort finden die Crossing-overs mit erhöhter Wahrscheinlichkeit statt. 

\begin{itemize}
    \item SNPs im gleichen Haploblock haben tendenziell hohes paarweises LD (werden häufiger gemeinsam vererbt, statistisch \textbf{abhängig} voneinander)
    \item SNPs in unterschiedlichen Haploblöcken haben tendenziell niedriges LD (werden seltener gemeinsam vererbt, statistisch \textbf{unabhängig} voneinander) 
\end{itemize}
\end{frame}


\begin{frame}{Aufgabe 1: Lösung (3)}
\begin{figure}[h]
\begin{center}
\includegraphics[width=0.8\textwidth]{../figures/Slides11_Haplotypes.jpg}
\caption{LD-Struktur des Genoms.}
\label{fig:Haploblock}
\end{center}
\end{figure}
\end{frame}

\begin{frame}{Aufgabe 1: Crossing-over \& LD (3)}
\begin{table}[h]
\caption{4-Felder-Tafel der beiden biallelischen SNPs: SNP 1 (Allele A1/B1) und SNP 2 (Allele A2/B2) aus Daten von 500 gemessenen diploiden Individuen} \label{tab:4FT} 
\begin{center}
\begin{tabular}[h]{c|cc}
 & SNP 1 - Allel A1 & SNP 1 - Allel B1 \\
\hline
SNP 2 - Allel A2 & 570 & 15 \\
SNP 2 - Allel B2 & 25 & 390 \\
\end{tabular}
\end{center}
\end{table} 

\begin{align*}
r^2 &= \frac{(p_{00}p_{11} - p_{01}p_{10})^2}{p_{0.}p_{.0}p_{1.}p_{.1}} 
\end{align*}

\end{frame}

\begin{frame}{Aufgabe 1: Lösung (4)}
\begin{table}[h]
\caption{4-Felder-Tafel der beiden biallelischen SNPs: SNP 1 (Allele A1/B1) und SNP 2 (Allele A2/B2) aus Daten von 500 gemessenen diploiden Individuen} \label{tab:4FT_L} 
\begin{center}
\begin{tabular}[h]{c|cc|c}
 & Allel A1 & Allel B1 & Randverteilung \\
\hline
Allel A2 & $p_{00}$=570 & $p_{01}$=15 & $p_{0.}$=585 \\
Allel B2 & $p_{10}$=25 & $p_{11}$=390 & $p_{1.}$=415 \\
\hline
Randverteilung & $p_{.0}$=595 & $p_{.1}$=405 & 1000
\end{tabular}
\end{center}
\end{table}

\begin{align*}
r^2 &= \frac{(p_{00}p_{11} - p_{01}p_{10})^2}{p_{0.}p_{.0}p_{1.}p_{.1}}  \\
 &= \frac{570 \cdot 390 - 25 \cdot 15)^2}{585 \cdot 595 \cdot 415 \cdot 405}  \\
 &= 0.842 \notag \\
\end{align*}
\end{frame}

\begin{frame}{Aufgabe 1: Lösung (5)}
\begin{itemize}
    \item Hohes LD zwischen SNP1 und SNP2, nicht statistisch unabhängig! 
    \item Häufige Haplotypen: A1A2 und B1B2
    \item Selten Haplotypen: A1B2 und B1A2
    \item Erwartung: niedrigeres LD zwischen SNP 1 und 3
\end{itemize}
\end{frame}

\begin{frame}{Zusatz zu LD (1)}
Was ist LD und warum ist es wichtig in der Funktionelle Genomanalysen? 
\end{frame}

\begin{frame}{Zusatz zu LD (2)}
\begin{itemize}
    \item Allele sind in LD wenn sie häufiger gemeinsam vorkommen als zufällig erwartet (Korrelation)
    \item $r^2$ ist ein häufig genutztes Maß für LD 
    \item $r^2$ ist zwischen 0 (kein LD) and 1 (perfekte Korrelation).
    \item LD kann von verschiedenen Faktoren beeinflusst werden: 
    \begin{itemize}
        \item Mutationsrate
        \item Genetischer Drift
        \item Nicht-zufällige Paarung (Zucht)
        \item Populationsstruktur
        \item Selektion
        \item Genkopplung
        \item Rekombinationsrate
    \end{itemize}
\end{itemize}
\end{frame}

\begin{frame}{Zusatz zu LD (3)}
\begin{figure}[h]
\begin{center}
\includegraphics[width=0.8\textwidth]{../figures/Slides11_LD_1.jpg}
\caption{Recombination between unlinked loci (= not in LD).}
\label{fig:LD1}
\end{center}
\end{figure}
\end{frame}

\begin{frame}{Zusatz zu LD (4)}
\begin{figure}[h]
\begin{center}
\includegraphics[width=0.8\textwidth]{../figures/Slides11_LD_2.jpg}
\caption{Recombination between linked loci (= in LD).}
\label{fig:LD2}
\end{center}
\end{figure}
\end{frame}

\section{Exercise 2}

\begin{frame}{Aufgabe 2: HWE (1)}
Für den biallelischen SNP 1 mit Allelen A/B wird folgende Genotypverteilung beobachtet:

\begin{table}[here]
\begin{center}
\begin{tabular}[h]{ccccc}
Genotyp & AA & AB & BB & Missing \\
\hline
Häufigkeit & 824 & 1326 & 463 & 87\\
\end{tabular}
\end{center}
\end{table}

\begin{enumerate}
    \item Welche Modellannahmen werden Hardy-Weinberg-Gleichgewicht (HWE) getroffen (Stichwort \textbf{ideale Population})?
    \item Bestimmen Sie auf drei Nachkommastellen genau die 
    \begin{enumerate}
        \item \textbf{Callrate} des SNPs,
        \item \textbf{Allelfrequenzen} für A und B, und 
        \item \textbf{erwartete Genotypverteilung} im HWE!
    \end{enumerate}
\end{enumerate}
\end{frame}

\begin{frame}{Aufgabe 2: Lösung (1)}
\textbf{Ideale Population}:

\begin{itemize}
    \item Diploide Organismen
    \item Nur geschlechtliche Vermehrung
    \item Keine Überlappung der Generationen
    \item Zufällige Paarungen
    \item Unendliche Populationsgröße
    \item Allelfrequenzen sind in beiden Geschlechtern gleich
    \item Keine Migration, Gendrift, Mutation oder Selektion
\end{itemize}
\end{frame}

\begin{frame}{Aufgabe 2: Lösung (2)}
\textbf{Callrate}: 

$$CR =\frac{N_1}{N_0} = \frac{824 + 1326 + 463}{824 + 1326 + 463 + 87} = \frac{2613}{2700}=0.968$$

\textbf{Allelfrequenzen}: 

$$p = AF_A = \frac{2 \cdot AA + AB}{2 \cdot N_1} = \frac{2 \cdot 824 + 1326}{2 \cdot 2613} = 0.569$$

$$q = AF_B = \frac{2 \cdot BB + AB}{2 \cdot N_1} = \frac{2 \cdot 463 + 1326}{2 \cdot 2613} = 0.431$$

\end{frame}

\begin{frame}{Aufgabe 2: Lösung (3)}
\textbf{Erwartete Genotypverteilung}: Im HWE gilt: 

$$1 = p+q = (p+q)^2 = p^2 + 2pq + q^2$$

\begin{table}[h]
\begin{center}
\begin{tabular}[h]{ccccc}
Genotyp & AA & AB & BB & Missing \\
\hline
Häufigkeit & 824 & 1326 & 463 & 87\\
\hline
$p_{obs}$ & 0.315 & 0.507 & 0.177 & \\
\hline
$p_{exp}$ & $p^2$=0.324 & $2pq$=0.490 & $q^2$=0.186 &  \\
\end{tabular}
\end{center}
\end{table}
\end{frame}

\begin{frame}{Aufgabe 2: HWE (2)}

\textbf{Zusatz}: Testen Sie auf HWE mit 5\% Irrtumswahrscheinlichkeit. Stellen Sie dazu die \textbf{Nullhypothese} auf. Berechnen Sie die \textbf{Teststatistik} für diese und interpretieren Sie das Ergebnis. 

Formel: $$\sum_i\frac{(O_i - E_i)^2}{E_i}, i\in {AA, AB, BB}$$
  
\end{frame}

\begin{frame}{Aufgabe 2: Lösung (4)}
\textbf{Nullhypothese}: Die beobachteten Häufigkeiten der Genotypen befinden sich im Hardy-Weinberg-Gleichgewicht. 

Um die Hypothese zu testen, muss das $\chi^2$ bestimmt werden: 

$$ \chi^2 = N_1 \sum \frac{(p_{obs} - p_{exp})^2}{p_{exp}}= 3.1416$$
Da $m=2$ Allele betrachten werden, haben wir einen Freiheitsgrad: 

$$df = \frac{m(m-1)}{2}=\frac{2 \cdot 1}{2}=1 \rightarrow \chi_1^2=3.841$$

Das berechnete $\chi^2$ ist kleiner als die Schranke $\chi_1^2$, daher kann die Nullhypothese nicht verworfen werden. 

\end{frame}

\begin{frame}{Zusatz zu HWE (1)}
Warum ist HWE wichtig in der Funktionelle Genomanalysen?    
\end{frame}

\begin{frame}{Zusatz zu HWE (2)}
\begin{figure}[h]
\begin{center}
\includegraphics[width=1\textwidth]{../figures/Slides12_HWE.jpg}
\caption{Punnett square and De Finetti diagram.}
\label{fig:HWE}
\end{center}
\end{figure}
\end{frame}

\section{Exercise 3}

\begin{frame}{Aufgabe 3: Stammbäume (1)}
\begin{enumerate}
    \item Definieren Sie die Begriffe \textbf{dominant}, \textbf{rezessiv} und \textbf{Penetranz}.
    \item Betrachten Sie die drei Stammbäume und geben Sie folgendes an:
    \begin{itemize}
        \item eine Legende,
        \item die Träger/in,
        \item das wahrscheinlichstes Segregationsmuster mit Begründung
    \end{itemize}
\end{enumerate}
\end{frame}

\begin{frame}{Aufgabe 3: Lösung (1)}
\begin{itemize}
    \item \textbf{dominant}: eine Kopie des Risiko-Allels reicht aus, um den Phänotyp zu erzeugen
    \item \textbf{rezessiv}: nur wenn das Risiko-Allel homozygot vorliegt, kommt es zum Phänotypen
    \item \textbf{Penetranz}: WSK, mit der ein bestimmter Genotyp zur Ausbildung des zugehörigen Phänotyps führt
    \item Legende:
    \begin{itemize}
        \item Form = Geschlecht (Kreis: Frau; Quadrat: Mann)
        \item Füllung = Phänotyp/Krankheit (Keine Füllung: gesund; rote Füllung: erkrankt; ein Punkt im Kreis oder Quadrat: Träger/in)
    \end{itemize}
\end{itemize}
\end{frame}

\begin{frame}{Aufgabe 3: Plot A}
Bestimmen Sie den Erbgang des vorliegenden Stammbaums und den Genotyp aller Mitglieder!

\begin{figure}[h]
\begin{center}
\includegraphics[width=0.7\textwidth]{../figures/Slides13_FamilyTree_A1.jpg}
\caption{Stammbaum A.}
\label{fig:tree_A1}
\end{center}
\end{figure}    
\end{frame}

\begin{frame}{Aufgabe 3: Plot A - Lösung}
Lösung: autosomal-rezessiv, weil

\begin{itemize}
    \item Beide Geschlechter betroffen
    \item Generationen können übersprungen werden  
\end{itemize}

\begin{figure}[h]
\begin{center}
\includegraphics[width=0.7\textwidth]{../figures/Slides13_FamilyTree_A2.jpg}
\caption{Stammbaum A - Lösung.}
\label{fig:tree_A2}
\end{center}
\end{figure}    
\end{frame}

\begin{frame}{Aufgabe 3: Plot B}
Bestimmen Sie den Erbgang des vorliegenden Stammbaums und den Genotyp aller Mitglieder!

\begin{figure}[h]
\begin{center}
\includegraphics[width=0.7\textwidth]{../figures/Slides13_FamilyTree_B1.jpg}
\caption{Stammbaum B.}
\label{fig:tree_B1}
\end{center}
\end{figure}    
\end{frame}

\begin{frame}{Aufgabe 3: Plot B - Lösung}
Lösung: autosomal-dominant, weil

\begin{itemize}
    \item Beide Geschlechter betroffen
    \item jede Generation betroffen  
\end{itemize}

\begin{figure}[h]
\begin{center}
\includegraphics[width=0.7\textwidth]{../figures/Slides13_FamilyTree_B2.jpg}
\caption{Stammbaum B - Lösung.}
\label{fig:tree_B2}
\end{center}
\end{figure}    
\end{frame}

\begin{frame}{Aufgabe 3: Plot C}
Kinderwunsch in Generation III. Bestimmen Sie den Erbgang und den Genotypen der Mutter. Mit welcher Wahrscheinlichkeit werden die Kinder dieses Paares erkranken?

\begin{figure}[h]
\begin{center}
\includegraphics[width=0.8\textwidth]{../figures/Slides13_FamilyTree_C1.jpg}
\caption{Stammbaum C.}
\label{fig:tree_C1}
\end{center}
\end{figure}    
\end{frame}

\begin{frame}{Aufgabe 3: Plot C - Lösung}
Lösung: X-chromosomal rezessiv, weil

\begin{itemize}
    \item Deutlich mehr Männer betroffen
    \item Generationen können übersprungen werden  
\end{itemize}

Mutter hat 50\% Chance Trägerin zu sein

\begin{itemize}
    \item Töchter werden alle gesund sein (können Trägerinnen sein)
    \item Söhne werden zu 25\% erkranken (WSK(Mutter Trägerin) * WSK(rezessives Allel wird weitergegeben) = 0.5 * 0.5 = 0.25)
\end{itemize}

\begin{figure}[h]
\begin{center}
\includegraphics[width=0.5\textwidth]{../figures/Slides13_FamilyTree_C2.jpg}
\caption{Stammbaum C - Lösung.}
\label{fig:tree_C2}
\end{center}
\end{figure}    
\end{frame}

\section{Summary}

\begin{frame}{Zusammenfassung}

\begin{itemize}
    \item Warum ist LD wichtig in der Funktionelle Genomanalysen?
    \begin{itemize}
        \item (Un-)Abhängigkeit von genetischen Markern in statistischen Analysen!
    \end{itemize}
    \item Warum ist HWE wichtig in der Funktionelle Genomanalysen?
    \begin{itemize}
        \item Annahme einer best. Verteilung in stat. Analysen
        \item Kenntnis von best. Eigenschaften (z.B. Varianz)
    \end{itemize}
    \item Warum sind Segregationsmuster wichtig in der Funktionelle Genomanalysen?
    \begin{itemize}
        \item Relevant in Festlegung des Regressionsmodelles
    \end{itemize}
\end{itemize}
    
\end{frame}

\end{document}