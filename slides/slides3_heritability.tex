\documentclass{beamer}
\usetheme{Boadilla}
\usepackage{graphicx} % Required for inserting images
\usepackage{german}
\usepackage{hyperref}

\title{Funktionelle Genomanalysen 2023}
\subtitle{Asynchrone Übung zu Heritabilität}
\author{Dr. Janne Pott}
\date{09.-11. Juni 2023}

\begin{document}

\begin{frame}
\titlepage
\end{frame}

\begin{frame}{Definition}

\textbf{Heritabilität}: Anteil der Varianz eines Merkmals, der durch die Genetik erklärt wird. 

Beantwortet in wie fern Gene den Unterschied (Varianz) einer Eigenschaft erklären, \textbf{NICHT} welche Gene die Eigenschaft beeinflussen. 

$$ h^2 = \frac{Var(G)}{Var(Merkmal)} = \frac{Var(G)}{Var(G) + Var(U) + 2 \cdot Cov(G, U)}$$

\end{frame}

\begin{frame}{Methoden}
\begin{itemize}
    \item Verwandtschaftstudien bzw Zwillingsstudien $h^2$
    \begin{itemize}
        \item Falconers Formel $h^2=2 \cdot (r(MZ) - r(DZ))$ 
        \item Vergleich der Merkmalskonkordanz zwischen monozygoten (MZ) und dizygoten (DZ) Zwilligen
    \end{itemize}
    \item Genomweit via Querschnittsstudien von unverwandten Personen $h_{SNP}^2$
    \begin{itemize}
        \item Genetik-Daten vorhanden: GREML (z.B. in GCTA implementiert)
        \item Nur Summary Statistics vorhanden: LD Score Regression (python-basiert, bislang nur für weiße Europäer/Amerikaner etabliert)
    \end{itemize}
    \item SNP-basiert $h_{sSNP}^2$
    \begin{itemize}
        \item $r^2$ aus linearen Regressionsmodell ohne weitere Adjustierung
        \item Einzelschätzer pro SNP 
    \end{itemize}
\end{itemize}
\end{frame}

\begin{frame}{Missing heritability}

\begin{figure}[h]
\begin{center}
\includegraphics[width=1\textwidth]{../figures/Slides24_MissingHeritab.jpg}
\label{fig:Heritab}
\end{center}
\end{figure}

\end{frame}

\begin{frame}{Was Heritabilität ist (1/3)}
    \begin{itemize}
        \item Semi-formale Definition:
        \begin{itemize}
            \item Heritabilität ist der Anteil der Variabilität eines Merkmals, der durch Genetik erklärt werden kann. Es misst also inwieweit Unterschiede in der DNA Unterschiede in einem Merkmal erklären können.
            \item Heritabilität ist definiert zwischen 0 (Genetik erklärt nichts) und 1 (Genetik erklärt alles) 
            \item Beispiel Körpergröße: $h^2 \approx 0.8$ 
            \item Beispiel Schlafdauer: $h^2 \approx 0.15-0.2$
        \end{itemize}
        \item Heritabilität schätzt, wie gut wir ein Merkmal anhand der Genetik vorhersagen könnten
        \begin{itemize}
            \item wenn wir alle relevanten genetischen Auswirkungen vollständig verstanden hätten (was wir jedoch noch nicht haben!)
            \item quasi eine Obergrenze dafür, wie gut diese Vorhersage jemals sein könnte, wenn wir mehr über die Genetik des Merkmals erfahren.
        \end{itemize}
    \end{itemize}
\end{frame}

\begin{frame}{Was Heritabilität ist (2/3)}
    \begin{itemize}
        \item Heritabilität misst wie wichtig Genetik für ein Merkmal ist. 
        \begin{itemize}
            \item Hohe Heritabilität bedeutet \textbf{nicht} notwendigerweise dass es monogenetisch vererbt wird. 
            \item Hohe Heritabilität bedeutet, dass der Gesamtbeitrag direkter und indirekter kausaler Effekte und anderer Korrelationen zwischen bestimmten DNA-Varianten und dem Merkmal ausreichen, um informativ zu sein.
        \end{itemize}
        \item Heritabilität ist eine Eigenschaft einer Population, nicht eines Individuums.
        \begin{itemize}
            \item Heritabilität erklärt \textbf{nicht}, warum jemand eine Krankheit erleidet.
        \end{itemize}
    \end{itemize}
\end{frame}

\begin{frame}{Was Heritabilität ist (3/3)}
    \begin{itemize}
        \item Heritabilität ist abhängig von der Messmethode.
        \begin{itemize}
            \item Je schwieriger ein Merkmal zu messen ist, desto größer wird der zufällige Messfehler. Als Konsequenz wird die Heritabilität kleiner, da der Messfehler nicht genetisch ist. 
            \item Heritabilität ist abhängig davon, wer misst (Selbstbericht vs. gesicherte Diagnose vs. echte Messung)
            \item Heritabilität ist abhängig davon, wie man das Merkmal festlegt (Einnahme eines bestimmten Medikaments vs Diagnose)
        \end{itemize}
        \item Heritabilität ist abhängig von der Population.
        \begin{itemize}
            \item Heritabilität eines Merkmals in einer Population aus einer bestimmten Land, Ethnie, Alter, Sozioökonomischer Status, o.ä. kann sich von der aus einer anderen Population mit anderem genetischen Background unterscheiden.
        \end{itemize}
    \end{itemize}
\end{frame}

\begin{frame}{Was Heritabilität nicht ist (1/2)}
\begin{itemize}
    \item Heritabilität ist nicht Schicksal. 
    \begin{itemize}
        \item Nur weil ein Merkmal eine hohe Heritabilität aufweist und in deinen Eltern vorkommt, heißt das nicht, dass man selbst das Merkmal hat. 
        \item Es mag wahrscheinlicher sein, aber es ist nicht unvermeidlich.
    \end{itemize}
    \item Heritabilität misst nicht unsere Fähigkeit, das Merkmal zu beeinflussen.
    \begin{itemize}
        \item Haarfarbe ist hoch heritabel, aber jeder von uns kann sich die Haare färben wie er oder sie will (inklusive Farben die es in der Natur nicht gibt).
        \item BMI ist heritabel, aber durch Diät und Bewegung haben ebenfalls einen Einfluss auf BMI.
        \item Heritabilität ist nicht die finale Antwort auf “nature vs. nurture”.
    \end{itemize}
\end{itemize}
\end{frame}

\begin{frame}{Was Heritabilität nicht ist (2/2)}
\begin{itemize}
    \item Heritabilität ist nicht unveränderlich. 
    \begin{itemize}
        \item Änderung in der Umwelt wird eine Änderung in der Heritabilität erzeugen
    \end{itemize}
    \item Hohe Heritabilität bedeutet nicht, dass Gruppenunterschiede auf Genetik basieren.
    \begin{itemize}
        \item Historisch: Leider wurden “Rassenunterschiede“ bei den IQ-Werten auf die Genetik zurückgeführt
        \item \textbf{ABER}: Heritabilität hängt von der Messmethode, Population und Umgebung ab, und kann sich im Laufe der Zeit ändern!
        \item Es ist daher nicht zulässig, die geschätzte Heritabilität eines Merkmals als Beweis für „inhärente“ Unterschiede zwischen Populationen zu verwenden.
    \end{itemize}
\end{itemize}
\end{frame}

\begin{frame}{Additional Reading}
\begin{itemize}
    \item \href{http://www.nealelab.is/blog/2017/9/13/heritability-101-what-is-heritability}{Heritability 101} by Raymond Walters with contributions from Claire Churchhouse and Rosy Hosking
    \item For more academic discussion of pitfalls and misconceptions in understanding heritability: \href{https://www.nature.com/articles/nrg2322}{Visscher et al 2008 }
\end{itemize}
\end{frame}
\end{document}